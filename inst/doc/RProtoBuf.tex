\documentclass{article}

\usepackage{url}
\usepackage{Sweave}

\author{Romain Fran\c{c}ois, Dirk Eddelbuettel, Saptarshi Guha}
\title{RProtoBuf 0.0-4: R API for Protocol Buffers}

\begin{document}
\maketitle

\section{Protocol Buffers}

Protocol buffers are a language-neutral, platform-neutral, extensible 
way of serializing structured data for use in communications 
protocols, data storage, and more. 

It is an open-source language (BSD license) developed by Google. The 
protocol buffer project (\url{http://code.google.com/p/protobuf/})
contains a C++ library and a set of runtime libraries and compilers for
C++, java and python. 

With these languages, the workflow consists of 
compiling a protocol buffer description file into
language specific classes that can be used to create, read, write and 
manipulate protocol buffer messages. The project page contains a tutorial
for each of these officially supported languages:
\url{http://code.google.com/apis/protocolbuffers/docs/tutorials.html}

Aside C++, java and python implementations, several projects have been 
created to support protocol buffers for many languages. The list of knowns
languages to support protocol buffers is compiled as part of the
project page: \url{http://code.google.com/p/protobuf/wiki/ThirdPartyAddOns}

\subsection{Example}

Through this document, we will use the \texttt{addressbook} example
that is also used by the official tutorials for java, python and C++. 

\begin{Schunk}
\begin{Soutput}
// See README.txt for information and build instructions.

package tutorial;

option java_package = "com.example.tutorial";
option java_outer_classname = "AddressBookProtos";

message Person {
  required string name = 1;
  required int32 id = 2;        // Unique ID number for this person.
  optional string email = 3;

  enum PhoneType {
    MOBILE = 0;
    HOME = 1;
    WORK = 2;
  }

  message PhoneNumber {
    required string number = 1;
    optional PhoneType type = 2 [default = HOME];
  }

  repeated PhoneNumber phone = 4;
}

// Our address book file is just one of these.
message AddressBook {
  repeated Person person = 1;
}
\end{Soutput}
\end{Schunk}

\end{document}

